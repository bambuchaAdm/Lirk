\documentclass[11pt]{article}
\usepackage[utf8]{inputenc}
%\usepackage[T1]{fontenc}
\usepackage{amssymb}
\usepackage{amsmath}
\usepackage{enumerate}
\usepackage{fullpage}
\usepackage{polski} 
\usepackage{indentfirst} 
\usepackage[pdftex]{graphicx}
\usepackage{multirow}
\usepackage{placeins}
\usepackage{listings}
\usepackage{multicol}
\usepackage{hyperref}



\author{Łukasz Dubiel}
\title{LIRK \\ Lisp AVR with assembly Kernel}

\begin{document}

\maketitle

\section{Wprowadzenie}

Na platformie AVR istnieje już kilka języków mających ugruntowaną pozycję oraz zestaw narzędzi. Kilka głównych z nich to C, Bascom czy Pascal. Większość z nich to języki niezależne od platformy. Jednak mając takie ,,zabawki'' jakimi są mikrokontrolery z rodziny AVR, języki które operują na pełnych słowach stają się niewygodne. Dlatego postanowiłem spróbować innego podejścia do problemu.

\subsection{Lisp}

W roku 1958 John McCarthy na MIT stworzył język programowania, który został nazwany Lispem. Był to drugi, zaraz po FORTRANie język wysokiego poziomu. Przez ten czas Lisp ewoluował, jak również pojawiały się różne jego dialekty (oczywiście nie kompatybilne z sobą), aż stworzyły całą rodzinę języków. Obecnie gdy używa się słowa Lisp to mówi się o całej rodzinie języków lub o Common Lispie, dialekcie ustandaryzowanym przez ANSI w 1994 z późniejszymi modyfikacjami.

\subsection{Motywacje}

Z racji mojego dużego zainteresowania językami z rodziny Lispów oraz problemów jakie napotyka C w momencie, gdy trzeba operować na pojedynczych bitach, a nie słowach, zdecydowałem się na stworzenie "dialektu" Lispa na mikrokontrolery AVR. 

Początkowo miało to być ćwiczenie w celu lepszego poznania Common Lispa, lecz potem zdecydowałem się, że to właśnie jego będę wykorzystywał w moich hobbystycznych projektach.

\section{Zamierzenia i cele}
Wposmniany wcześniej nowy dialek, Lirk,  nie ma (na razie) ambicji stania się nowym językiem programowania. Jest on zestawem makr, funkcji oraz plików które pozwalają na pisanie w assembly z lispową składnia jako cześć wykonywalna na mikrokontrolerów. Głównym celem jest udostępnienie funkcji, które zminimalizuje ilość "niepotrzebnego" kodu w konstrukcjach stricte assemblerowych.

\subsection{Ewaluowanie parametrów instrukcji}

Do działania Lispa faktycznie potrzebne są dwa byty - Reader i Evaluator. Pierwszy z nich ma za zadanie sparsować dane oraz rozwinąć tzw reader-macra (proste aliasy na najczęściej spotykane konstrukty językowe takie jak $'$ na quote czy $\#'$ na function), oraz wszystkie nazwy zmieni na duże litery (Lisp jest insensitive). Drugi z nich - evaluator - ma za zadanie wyznaczyć wartość wyrażenia które stworzył reader. 

Daje to duże możliwości (przykładem jest system makr lispowych), , które możemy wykorzystać by wzmocnić ekspresywność assembly. 

\lstinputlisting[language=Lisp]{agr.lisp}

Rozwiązanie jest podobne do zagnieżdżaniu funkcji w C jednak tam się to dzieje w momencie wykonania, a Likr czyni to podczas generowania pliku wykonywalnego.
Oczywistym jest, że wartość zwracana przez procedurę nie może być dynamiczna i zależeć od środowiska (stanu pinów czy innych).
Jednakże nie stanowi to jakiegokolwiek problemu, gdyż w procedurze możemy zapisać wartość do rejestru i takowy rejest zwrócić z procedury.

\subsection{Definicje oraz wołanie procedur}

Jednym z pierwszych problemów jakie nękały programistów była duplikacja kodu. Przeprowadzono badania wskazujące na to, że średnia ilość błędów na jednostkę kodu (KLOC czyli tysiąc linii) jest stała i zależy do wielkości projektu. Z faktu, że assembler jest jednym najmniej ekspresywnywnych języków (ilość kodu potrzebna do wykonania zamierzonej akcji jest duża) jest bardzo podatny na duplikacje.

Oprócz niskopoziomowych aspektów programowania interesuję się bardziej ogołnymi technikami jak i metodykami. Bardzo spodobał się mi manifest programowania zwinnego (ang. agile) oraz późniejszy, chociaż bardziej ważny dla programistów manifest Software Craftsmanship, , którego jednej z głównych haseł jest "Don't repeat yourself." Hasło bardzo ważne, gdyż powtarzanie kodu jest jedną z głownych przyczyn bugów. 

Jak widać "klasyczne" assembly, ze względu na swoją ekspresywność, stoi w kontraście do programów tych manifestów. 

Na tak małej platformnie jaką jest AVR bardzo dobrze sprawuje się programowanie proceduralne. Na podobnych platformach działało C 30 lat temu. Lekko problematyczne w C jest to iż, każda procedura działa na zasadzie skoku i powrotu (standard C99 zaczerpnął z C++ słowo kluczowe inline, jednak sprawia on, że funkcja nie zawsze jest rozwijana w miejscu i zależy to od wielu czynników). 

Lirk z zamierzeniu ma oferować dla programisty obie te metody. Procedura może być wywołana przez skok z odłożeniem adresu powrotu na stos, jak i być rozwinięta w miejscu.
\newpage

\begin{multicols}{2}
\lstinputlisting[language=Lisp]{inplace.lisp}
\columnbreak
\lstinputlisting{inplace.asm}
\end{multicols}
albo wywołać jak wykonując skok
\begin{multicols}{2}
\lstinputlisting[language=Lisp]{jump.lisp}
\columnbreak
\lstinputlisting{jump.asm}
\end{multicols}

Oczywiście w tym przypadku kompilator automatycznie zarządza etykietami oraz dodanie odpowiedni fragment kodu z instrukcjami powrotu.

\subsection{Przerwania}

Na przerwania można popatrzyć jak na specyficznie definiowane procedury. Procedury, które wyzwalane są zdarzeniami generowanymi przez systemy mikrokontrolera.
Wiadome jest że na samym początku przestrzeni programu znajduje się tablica wektorów przerwań.

Skoro jest to specyficzna funkcja możemy sobie stworzyć makro, które ułatwi nam jej definiowanie

\lstinputlisting[language=Lisp]{macro.lisp}


Dzięki niemu możemy tworzyć przerwania w następujący sposób.

\lstinputlisting[language=Lisp]{interrupt.lisp}

\subsection{Tworzenie rozgałęzień sterowania}
Asembler oferuje kilkanaście instrukcji warunkowych. Są one w postaci predykatów. Język predykatowy jest naturalnym sposobem tworzenia warunków w językach funkcyjnych. 

\begin{multicols}{3}
\lstinputlisting[language=C]{predykat_1.c}
\columnbreak
\lstinputlisting[language=Lisp]{predykat_1.lirk}
\columnbreak
\lstinputlisting{predykat_1.asm}
\end{multicols}

Widać, że kod w Common Lispie (, którego składnie przyejmuje Lirk) wygląda inaczej niż analogiczny kod w C. Różniece nie są duże i bardzo łatwo je przeskoczyć. Dodatkowo podobnie jak w Pythonie a nawet bardziej kod programu przypomina (z drobnymi modyfikacjami) tekst pisany w języku naturalnym opisujący sposób działania. Jednak z racji użycia jako podstawowych funkcji assemblera jest to lekko zaciemnione. Rozwiązaniem jest dodanie aliasów do poszczególnych instrukcji. 

Utrzymywanie bardzo rozgałęzionego kodu w assembly wymaga bardzo dużego skupienia oraz bardzo rozległego modelu mentalnego. W przypadku assembly dużo więcej jest w głowie programisty niż jest napisane.
Lirk ma na celu lekkie odzielenie oraz osłabienie wymagań skupienia programisty oraz większe zgrupowania akcji powiązanych z sobą w logiczną całość.

Rozwiązania oparte na predykatach (, które w większości stosuje programowanie funkcyjne) bardzo dobrze mapuje się na instrukcje rozgałęzienia sterowania w kodzie,
oraz zmniejsza ryzyko popełnienie błędu przez programistę (brak drugiego znaku równości podczas porównywania)

\subsubsection{Dalsze plany}

W planach jest dodanie możliwości, że raz wołana funkcja jest rozwijana w miejscu, bez dodania niepotrzebnych instrukcji i etykiet, ale jeżeli programista wywoła procedurę powtórnie, w assembly pojawiają się wszystkie potrzebne elementy by wykonać odpowiednie skoki.

Rozwiązanie umożliwia większą granulację kodu, oraz ułatwia testowanie jednostkowe. Nawet można się pokusić o użycie TDD\footnote{\url{http://www.agiledata.org/essays/tdd.html}} jednak wymaga to napisanie odpowiednich narzędzi (emulatorów platformy,narzędzi do budowania, framework testowy), które nie stanowią dużego problemu (przynajmniej w ograniczonej formie).


\section{Narzędzia}
Obecnie Lirk nie posiada żadnych narzędzi, jednak w raz z rozwojem to się zmieni.
\subsection{IDE}
W pierwszych fazach rozwoju Lirka jako zintegrowane środowisko prograqmistyczne używany będzie GNU Emacs. 
Oczywiście trzeba będzie dodać główny tryb działania, który będzie ułatwiał tworzenie w Lirku.
Emacs posiada już dwa tryby dla Lispów (Emacs Lisp oraz dla innych lispów), tak więc tryb dla Lirka będzie modyfikacją instniejącego.
\subsection{Budowanie}
Oczywistym jest że obecnie nie mam dedykowanego narzędzia do budowania projektów tworzonych w  Lirku.
W tej roli wykorzystywany jest GNU Make, który wywołuje intereter lispa z odpowiednimi argumentami.

Sądzę, że stowrzenie własnego systemu do budowania ułatwi towrzenie oprogramowanie w Lirku 
jak i integrację z instniejącymi zintegrowanymi środowiskami programistycznymi oraz programatorami
czy innymi narzędziami takimi jak 

Gdy korzysta się z TDD oczywistym jest że będziemy potrzebować narzędzia do tworzenia testów, ich wykonywania jak i zbierania informacji.

Jako środowisko testowe można wykorzystać projekt simavr\footnote{\url{https://gitorious.org/simavr#more}} jako emulatora platformy AVR na procesorach x86. 
Implementuje on znaczną część funkcjonalności znacznej dużej cześci mikrokontrolerów AVR.
Umożliwi łatwe badanie stanu rejestrów, dzięki czemu wykonywanie testów jednostkowych stanie się możliwe oraz
nie będzie potrzebne ładowanie programu na fizyczny mikrokontroler, który posiada ograniczoną ilość zapisów.

Do procesu TDD oprócz środowiska potrzebny jest mechanizm definiowania testów.
W tym miejscu potrzebna będzie biblioteka, która pozwoli na łatwe ustalanie stanu początkowego,
definicja samego testu.
Oprócz samego przeprowadzania akcji testowych potrzebne jest system asercji wartości rejestrów oraz stanu pinów,
który pozowli na weryfikowanie czy kod dokładnie wykonuje akcje które chcemy.
Kod będzie można dokładnie testować ze względu na jego zachowanie,
dzięki czemu odstęp czasowy między wprowadzeniem buga do kodu,
a zdiagnozowaniem jego istnienia oraz jego usunięciem bardzo się skraca.
Zaoszczędzi to czasu na debugowanie, które jest trudną operacją podczas pisania w assemblerze.

\subsection{Pliki specyficzne dla sprzętu}
W assmeblerze istnieją pliki definiujące offsety poszczególnych rejestrów oraz stałe dla konkretnego modelu mikroprocesora.
By móc zapewnić polimorfizm ze względu na rejestr, Lirk, również musi posiadać takowy plik.
Zawierałby podobne dane do jego assemblerowego odpowiednika lecz opakowane w CLOS\footnote{\url{http://www.dreamsongs.com/NewFiles/ECOOP.pdf}}
 
Posiadanie takowego pliku jest również podyktowane tym, że gdzieś w programie musi następować
dołączenie do wynikowego kodu assembleroweogo dyrektywa dołączająca plik urządzenia.
Tak więc plik z definicjami dla Lirka musiałby istnieć. Można go rozszerzyć o kilka stałych oraz rzeczy zależnych od modelu.
Uprości to towrzenie Lirka oraz usunie niepotrzebną komplikację jego.

\section{Podsumowanie}
Obecnie Lirk realizuje tylko fragment z opisanych funkcji, ale trwają nad tym prace. 

Jedynym ograniczeniem jest tu brak czasu (zajęcia na uczelni, oraz inne projekty i zobowiązania). Jednak mimo ograniczeń, Lirk, powoli, ale się rozwija.

\end{document}
